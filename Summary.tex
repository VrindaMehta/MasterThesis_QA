\documentclass[main.tex]{subfiles}
\begin{document}
%\chapter{Summary}\label{ch:Summary}
To solve a SAT-problem by means of quantum annealing, the problem needs to be mapped to the Ising Hamiltonian. The idea of quantum annealing is exploring the energy spectrum of the resulting Hamiltonian by making use of quantum fluctuations induced by the transverse magnetic field in the model. In this work, efforts were made to simulate a quantum annealer to solve 2-SAT problems for a set of 8-spin and 12-spin problems. This requires the time dependent Schr{\"o}dinger equation to be solved, to obtain the final state of the system under the action of the Hamiltonian. In addition to the two terms commonly used by D-wave for quantum annealing (the initial Hamiltonian and the problem Hamiltonian), a third term, namely the trigger Hamiltonian, was introduced.  The objective of this work was to study the effects of including the trigger Hamiltonian on the performance of quantum annealing.\\
The set of 12-spin 2-SAT problems contained 1000 problems, while that for 8-spin 2-SAT contained 91 problems. There were two types of trigger Hamiltonians employed: a ferromagnetic trigger and an anti-ferromagnetic trigger. Additionally, the strength with which the trigger was added was controlled by means of a variable $g$, which was chosen to have values 0.5, 1 and 2, while the annealing time, $T_A$, was chosen to be 10, 100 or 1000.\\

For the 12-spin problems, it was observed that compared to the minimum energy gaps between the ground state and the first excited state of the original Ising Hamiltonian (i.e., in the absence of any triggers), the minimum energy gaps after adding the ferromagnetic trigger were enlarged for all the 1000 problems of the set, for all the three values of $g$. Furthermore, for every problem the enlargement of the minimum energy gaps was directly proportional to the strength with which the trigger was added. Consequently, the success probabilities after adding the ferromagnetic trigger for a given $T_A$, were larger than the original success probabilities for the corresponding $T_A$. This suggests that the evolution of the system state is close to adiabatic in this case, and increasing the strength of the ferromagnetic trigger and the annealing time lead to larger success probabilities.\\

On the other hand, the effects of adding the antiferromagnetic trigger depend strongly on the strength parameter $g$, and the chosen annealing time $T_A$, but more so on the problem at hand. For a majority of the problems, the minimum energy gap decreased as a result of adding the anti-ferromagnetic trigger (999 problems for $g$=0.5, 879 for $g$=1, and 798 for $g$=2). It was also observed that irrespective of the strength with which the anti-ferromagnetic trigger was added, and the chosen annealing time, the success probability of the majority of problems decreased upon including the trigger. Additionally, for larger values of $g$, the number of anti-crossings between the ground state and the first excited state also increased. For 12-spin problems, and $g$=1, a majority of problems were found to have 2 anti-crossings, while for $g=2$ it increased to 3. For $g$=2, only a very few problems had a larger success probability for $T_A$=10, while the number increased for $T_A$=100 and $T_A$=1000. On the other hand, for $g$=0.5 and $g$=1, the maximum number of cases with improved success probability upon adding the trigger, corresponded to $T_A$=10. On increasing the annealing time to 100 and 1000, the number of problems with improved success probability was found to drop in both these cases. This suggests that in most of these cases the success probability increases due to the non-adiabatic evolution of the state. Some of these mechanisms were observed by studying the energy spectrum with instantaneous energy expectation values, or by computing the instantaneous overlap of the system state with a few low lying energy states of the Hamiltonian. It was noted that choosing annealing times small enough for the system state to deviate from the ground state before the anti-crossing, so that some of the amplitude comes back at the anti-crossing, or presence of multiple anti-crossings with small and comparable energy gaps, or a large deviation of the state from the ground state to end in a superposition state with larger overlap with the ground state, can turn out to be advantageous for the success probability.\\





\end{document}