\documentclass[12]{article}

\begin{document}
\textbf{Abstract}\\
Quantum annealing is one of the two standard approaches for devising a quantum computer, where the computation advances from an initial Hamiltonian whose ground state is easy to prepare, to a final Hamiltonian whose ground state encodes the solution to an optimization problem. The aim of this work was to simulate a quantum annealer for 12-Boolean-variable 2-SAT problems having a unique ground state, and a highly degenerate first excited state, and to study the effects of introducing a third Hamiltonian, namely, a trigger Hamiltonian, which vanishes at both the start and the end of the annealing process. For this, two types of trigger Hamiltonians - a ferromagnetic Hamiltonian and anti-ferromagnetic Hamiltonian, were employed. It was found that adding the ferromagnetic trigger always enhanced the performance of quantum annealing by increasing the minimum energy gap between the ground state and the first excited state of the total Hamiltonian. Although adding the anti-ferromagnetic trigger reduced the minimum energy gap for a majority of the problems, the performance was improved for a fraction of the problems as a result of an increase in the number of anti-crossings between the two lowest lying states.

\end{document}